\chapter{Problem posing}

\section{Manual conversion to QUBO form}

To utilize the quantum computing systems provided by D-Wave, the formulation of SVMs must be slightly modified. These minor adjustments are necessary to transform a CSP into a QUBO form.

First of all, it is essential to transform the objective function into a minimization problem due to the physical nature of how quantum machines seek solutions. To transform the current objective function, it is sufficient to invert the signs of each component in the sum because the optimal solution for $\max f(x)$ is equivalent to the best solution for $\min -f(x)$, resulting in:

\begin{equation}\label{eq:min-svm}
    \min_\alpha\ \frac{1}{2}\sum_{i=1}^n\sum_{j=1}^n\alpha_i\alpha_jy_iy_jK(x_i, x_j) - \sum_{i=1}^n\alpha_i
\end{equation}

Constraint \ref{eq:svm-c2} can be incorporated into the objective function as a penalty constraint through Lagrangian relaxation. This procedure simplifies the finding of potential solutions since the problem’s constraints no longer need to be explicitly satisfied. However, by appropriately calibrating the Lagrangian multiplier, it is possible to ensure that the optimal solutions are those that satisfy the constraints, effectively reducing the penalty constraints to zero.

$$\min_\alpha\ \frac{1}{2}\sum_{i=1}^n\sum_{j=1}^n\alpha_i\alpha_jy_iy_jK(x_i, x_j) - \sum_{i=1}^n\alpha_i + L\left(\sum_{i=1}^ny_i\alpha_i\right)^2$$

An empirical method advises setting the Lagrangian parameter $L$ to a value between 70\% and 150\% of a bound of the solution calculated greedily\cite{QbridgeI}.

The final step to achieve a QUBO formulation is to rewrite the problem using binary variables. The presented SVM formulation uses the real domain for the variables $\alpha$, making them not directly usable. A ``naive'' solution to this problem could be to use a standard encoding for real numbers, similar to traditional computers, such as IEEE 754\cite{IEEE}. However, since each binary variable corresponds to a node in the QPU graph, this approach would be highly inefficient, requiring a large number of qubits that would essentially be underutilized.

Alternatively, ad hoc representations can be evaluated to minimize the number of qubits to the bare essentials. Using the information from constraint \ref{eq:svm-c1}, which limits the domain of the variables $\alpha$ to positive numbers less than $C$, can be helpful. Additionally, machine learning models often do not require the full numerical precision available in modern computers, making it reasonable to consider limiting the precision of the expressible real numbers to a fixed number of decimal places, further reducing the number of qubits needed for a single $\alpha$.

A final possible refinement could be to use a numerical representation system based on powers of ten instead of powers of two. For instance, the four-bit number $1011$ could represent $10.11$ ($10^1 + 10^{-1} + 10^{-2}$) instead of $2.75$ ($2^1 + 2^{-1} + 2^{-2}$), in this representation system, the number of bits is divided equally between the integer and fractional parts.

The choice of the Lagrangian hyperparameter and the bit system to represent the optimization variables depends on the specific problem and the data considered.

\section{Automatic D-Wave conversion}\label{sec:dwaveconversion}

The previously described procedure can be performed automatically by the libraries provided by D-Wave. The products available through cloud computing relieve the developer of the responsibility of handling problems by imagining the qubit grid and instead allow writing only the mathematical formulation of the CSP, delegating the solver to transform and optimize the problem using the QPU.

This process is not done by directly accessing the quantum solver but through what are defined as hybrid solvers\cite{hybrid}. Intuitively, a hybrid solver prepares the problem on the QPU and divides the work between the QPU and CPU to speed up the optimization process by offloading the CPU from the more complex procedures.

Hybrid solvers can read the standard format in which optimization problems are written, allowing easier substitution of classical solvers with quantum solvers. For example, any optimization model written with Pyomo\cite{pyomo}, a standard library in combinatorial optimization, can save models in text files that can later be read and loaded into D-Wave’s internal solver format.

\paragraph{Domain limitations}\label{sec:domain} Reading the file describing the problem and converting it to the internal representation is entirely transparent as long as two conditions are met:
\begin{enumerate}
    \item The problem must be written in minimization form,
    \item There must be no multiplications between real variables.
\end{enumerate}
Equation \ref{eq:min-svm} shows that it is not a problem to meet the first requirement. However, it is different regarding the multiplication of optimization variables in the real domain. The SVM formulation involves quadratic interactions between variables in the real domain. To approximate the solution without requiring a complete problem rewrite, it is possible to convert all variables to the integer domain. In this way, using the information from constraint \ref{eq:svm-c1}, $\lceil\log_2(C)\rceil$ qubits are sufficient for each variable $\alpha$.

Empirically, the proposed approximation results in negligible performance deterioration while significantly reducing the number of qubits required for each variable.

\subsection{Pre-solving}\label{sec:presolve}

The guidelines provided by D-Wave to make best use of its hybrid solvers suggest performing the pre-solution operation for constrained quadratic models (CQM). This procedure aims to statically analyse the problem and eliminate redundant operations or restructure the problem to make the best use of available information.

While it is not always possible to guarantee optimal behaviour, pre-solving generally brings significant advantages during model resolution. Various strategies have been proposed in the literature, and state-of-the-art classical solvers autonomously implement similar policies to make the search more efficient.

Among the various techniques available, D-Wave implements some low-computational-cost strategies, all derived from \cite{PRESOLVE}.

\paragraph{Removal of redundant constraints} The removal of constraints is based on the structural limits of the problem. For instance, during the creation of SVMs, the variables $\alpha$ are declared as integers in $[0, C]$. Hence, a constraint like $\alpha_0 \geq 0$ is irrelevant since the information is contained in the domain of $\alpha$.

\paragraph{Removal of low magnitude bias} This operation is performed on both constraints and the objective function. Each solver has a parameter indicating the maximum difference to tolerate to consider two floating-point numbers equal, often called ``feasibility tolerance''. Some biases significantly below the ``feasibility tolerance'' can be rounded to zero since it is verified that they will never have significant impacts on the rest of the problem.

\paragraph{Domain propagation} Based on the problem constraints, the domain of the variables is reduced. This procedure not only limits the solutions but also has a desirable impact on the number of qubits required to express the problem variables. By reducing the domain, fewer nodes of the Pegasus graph are sufficient.

\paragraph{Note on pre-solving efficiency} The pre-solving strategy proposed by D-Wave has proven to have minimal impact in terms of computational overhead. A brief exploratory investigation showed that the number of nodes in the transformed problem could grow to double the original graph size. This behaviour could motivate specific cases where, although applicable, it might be optimal not to apply pre-solving to avoid excessive growth in the problem size.

In the specific case addressed, the problem dimensions are well below the system’s limits\cite{hybrid2}. Therefore, it is possible to take advantage of the pre-solving strategy.
