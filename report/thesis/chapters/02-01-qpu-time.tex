\chapter{Analysis of Hybrid Solving Times}

Section \ref{sec:qsvm-res} has shown that adiabatic quantum computing can offer advantages in terms of faster convergence to a solution. However, it is worth asking how much of this advantage is due to the quantum component and how much is attributable to a classical solver optimised for quadratic problems.

The technology behind the hybrid solvers is protected by D-Wave's intellectual property, which makes these tools essentially black boxes. Given the proprietary nature of these solvers, analysis must rely on the limited data available to make informed evaluations of alternative solutions.
From the control dashboard provided by D-Wave\footnote{\url{https://cloud.dwavesys.com/leap/}}, it is possible to inspect the problems solved by their QPUs and obtain some useful information.

Although the data collected does not allow an in-depth analysis of the underlying implementation choices, it can still be useful for a ``high-level'' analysis. By selecting a problem from the list, the following details are available: 
\begin{itemize} 
	\item A report on the characteristics of the problem presented, whether BQM, QUBO or CQM. 
	\item A list of the potential solutions calculated, together with the associated energy values, representing the result of the objective function with the given assignment. 
	\item Execution times of the problem, indicating not only the total time but also how much of this time was spent on the QPU. 
\end{itemize}

Of the available data, information on execution times is of main interest, as it can help determine whether and to what extent the quantum component is crucial in generating an optimal solution.

\section{QPU Usage}

Analysing the execution of the quantum support vector machine on a subset of TweetDF, it can be seen that of the 40 seconds it took to find the optimal solution, only 0.031 seconds were used by the QPU. Proportionally, the QPU was used for only 0.08\% of the total time required by the solver.

When using the entire TweetDF, the data indicated a QPU utilisation of 0.05 seconds, compared to the 480 seconds it took the solver to find a solution, reducing the QPU utilisation to 0.01\%.

These results highlight two fundamental aspects: 
\begin{itemize} 
	\item The use of the QPU is minimal compared to the classical component required to calculate the solution; 
	\item The use of the QPU seems to decrease as the problem size increases, while the time used by the classical component increases significantly. 
\end{itemize}

Considering that the D-Wave hybrid solver has proven to be more efficient than its classical counterpart, CPLEX, the question arises whether the performance improvement is due to the QPU or a superior classical solver.

As an alternative to D-Wave's hybrid approach, ``QPU-only'' solvers can be used. By manually performing the pre-processing steps, the classical reprocessing and resolution component can be avoided, while the entire calculation is performed on the QPU.