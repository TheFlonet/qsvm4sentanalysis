\section[Theoretical framework]{Theoretical framework}
\label{sec:starting_theory}

\subsection{Sentiment analysis}
\note{Cos'è la sentiment analysis, 
inquadrarla nell'ambito dell'elaborazione del linguaggio naturale, 
tecniche allo stato dell'arte,
rilevanza sul piano pratico,
collegamento con classificazione binaria}

\subsection{Support vector machine}
\note{Problema di generalizzazione nella classificazione binaria con ML,
intuizione della soluzione tramite SVM,
problema primale (training e inferenza),
derivazione del problema duale (training),
utilità del problema duale,
inferenza tramite il problema duale,
stato dell'arte sull'uso delle SVM (sklearn e L2-SVM)}

\subsection{Adiabatic quantum computing}
\note{Procedura di annealing in generale,
algoritmi classici che implementano annealing,
vantaggio del quantum annealing,
rappresentazione del problema (griglia di qubit),
formulazione del problema (QUBO)}