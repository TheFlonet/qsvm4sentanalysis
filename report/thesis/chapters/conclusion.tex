\chapter{Conclusion}

The research presented in this thesis has allowed for an in-depth exploration of:
\begin{enumerate}
    \item The effectiveness of adiabatic quantum computing when applied to natural language processing tasks;
    \item The study of the topological and architectural limitations of the current generation of quantum processing units (QPU) designed to implement adiabatic quantum computing;
    \item The development of hybrid solvers as alternatives to those proposed by D-Wave.
\end{enumerate}

\paragraph{Quantum Support Vector Machine} The study of machine learning models based on quantum technologies is a relatively new topic in the research landscape. 
Despite this, the potential to exploit quantum phenomena has piqued the interest of some researchers, who hope to achieve advantages in terms of performance, representation or both.

Among the works on quantum machine learning using quantum annealing, we can cite those by Vladislav Golyanik and his research group 4DQV\footnote{\url{https://4dqv.mpi-inf.mpg.de/}}, whose research focuses on the field of computer vision\cite{qcv1}\cite{qcv2}.

In natural language processing, there are no significant contributions aimed at integrating quantum annealing and language models. 
The experiments conducted as part of this research demonstrate that quantum computing can provide tangible benefits compared to traditional optimization methods, warranting further investigation. 
Additionally, it becomes clear that the trade-off between convergence speed and solution quality remains a variable to be assessed on a case-by-case basis. 
However, quantum-based models add to the range of available options.

In scenarios where computational resources are limited, quantum annealing processes can enable the development of useful models on personal computers and embedded systems. 
The solutions generated by quantum machines prove to be good approximations when compared to state-of-the-art models. 
Furthermore, given the reduction in model complexity, quantum machine learning techniques offer:
\begin{itemize}
    \item Improved interpretability of results, a significant factor in certain application contexts such as the medical field;
    \item Lower resource consumption, facilitating faster prototyping of new solutions.
\end{itemize}

\paragraph{QPU Analysis} The analysis of D-Wave's QPU utilisation strategies revealed that the computational resources required by the minor embedding algorithm do not allow problems with more than 128 logical qubits to be solved directly using the QPU. 
Beyond this empirical limit, the current Pegasus architecture lacks sufficient physical qubits and adequate interconnections between qubits to express the problems.

These limitations could be partially addressed with the introduction of new QPU topologies. 
For instance, Zephyr proposes a greater number of qubits and a graph with a generally higher degree.

However, even though newer QPUs might allow for the handling of increasingly large problems, the available resources are unlikely to be sufficiently extensive to meet the requirements for tackling real-world problem sizes. 

Even with the potential development of increasingly efficient algorithms for minor embedding and topological solutions that facilitate easier problem mapping onto QPUs, it is unlikely that the current trajectory of quantum annealing technologies will evolve without using hybrid paradigms. 
This makes the development of freely accessible solvers, both specialized for specific tasks and general purposes, a central necessity.

\paragraph{Homebrewing a Hybrid Solver} The hybrid solver proposed and developed during this thesis, \texttt{QSplitSampler}, has as its primary goal the extensive use of available quantum resources, seeking to rebalance the workload between the CPU and QPU.
Which in D-Wave's proposal seems to be heavily skewed towards classical CPU-intensive use.

The approach used by \texttt{QSplitSampler} allows application in various settings, far beyond what is possible with QML. 
This is due to the fact that the algorithm is based on the structure of the QUBO problem.
The main advantage of working with general problems such as those described by QUBO models is that the proposed procedure becomes easily reusable and adaptable across different contexts, beyond SVM optimization.

Currently, there are no guarantees regarding the practical utility of quantum computing in machine learning. 
Some studies focus specifically on verifying whether the procedures currently in use can be simulated efficiently on classical computers. 
Among these, we can mention the work of Marco Cerezo\cite{qcnn}, in which the paradigm of quantum convolutional neural networks, widely adopted in the literature, was shown not to exploit quantum phenomena appropriately, making the architecture classically simulatable.

The development of general-purpose solvers thus represents a more ``cautious'' path forward. 
Even if it were demonstrated that quantum computing cannot be effectively leveraged for machine learning tasks, solving QUBO problems remains an NP-complete challenge. 
Consequently, having a method to solve NP-complete problems quickly is still of significant research interest, encouraging an approach that remains agnostic to the specific application context.

\texttt{QSplitSampler} proposes an alternative solving method to those found in the literature, which include:
\begin{itemize}
    \item Iterative methods for dividing variables into subproblems as in Tameem Albash et al.\cite{subqubo2};
    \item Polyphase strategies to fix an increasing number of variable values, Dennis Willsch et al.\cite{subqubo1}.
\end{itemize}

\texttt{QSplitSampler} recursively partitions the QUBO problem, relying on algebraic properties, to maximize QPU usage. 
While this approach is reasonable, there is no guarantee increasing the use of the QPU will lead to better solutions.

The conducted tests suggest that purely algebraic decomposition may not be an optimal strategy for converging toward the global optimum. 
In Chapter \ref{sec:future}, some experimental approaches that could be explored to improve the performance of \texttt{QSplitSampler} will be discussed.