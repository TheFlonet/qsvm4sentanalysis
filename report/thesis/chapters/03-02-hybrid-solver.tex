\chapter{QUBO Splitting Algorithm}

Following the theoretical discussion on the feasibility of the algebraic partitioning method in Section \ref{sec:prop}, this chapter provides an account of its implementation, which is one of the outcome of the work. 
Chapter \ref{sec:qsplitres} will present the results obtained from testing the procedures outlined.

\section{QUBO Utility Class}

Thus far, QUBO problems have been presented and analyzed in matrix form or as the equivalent produced equation. 
While matrices can represent these problems, the coefficient matrix $Q$ is often sparse, i.e. it contains many zero values relative to the total number of matrix cells.

Since a QUBO problem is characterised by an upper triangular matrix, the maximum number of non-zero values is $n(n+1)/2$, where $n$ is the dimension of the $Q$ matrix. 
This implies, by definition, at least $n(n-1)/2$ zero values. Since QUBO problem information must be transmitted between different computers for processing, D-Wave has changed the representation format to eliminate redundant information.

To achieve this, instead of matrices, \textsc{HashMap}s are used, where the key is the pair representing the position in the matrix $(\text{row}, \text{col})$, and the value corresponds to the matrix coefficient at that cell. 
In this way, all zero values are ``discarded'', representing only the relevant information that characterizes the problem.

To operate effectively, it is necessary to maintain a dual representation of the QUBO problem: 
\begin{enumerate} 
	\item A matrix form for performing algebraic operations; 
	\item A minimal representation for transmitting the problem to D-Wave quantum solvers. 
\end{enumerate}

For this reason, a support class \texttt{QUBO} was developed during this thesis to manage the consistency between the two representations automatically. 
The \texttt{QUBO} class is responsible for: 
\begin{itemize} 
	\item Maintaining a copy of the matrix representation; 
	\item Maintaining a copy of the compact representation; 
	\item Storing a set of solutions for the problem, if provided; 
	\item Keeping track of the variables associated with the rows and col\-umns of the matrix. 
    This additional information is crucial because, during decomposition, the symbolic names of variables could be lost, leading to meaningless assignments.
\end{itemize}

The available methods are responsible for:
\begin{itemize} 
	\item Converting the problem from one representation to the other; 
	\item Verifying whether the matrix is upper triangular; if not, it is converted using the LU algorithm\cite{LU}.
\end{itemize}

\section{\texttt{QSplitSampler} Implementation}

The main points of the implementation are outlined below, along with an explanatory example; the complete code for \texttt{QSplitSampler} is available on GitHub\footnote{\url{https://github.com/TheFlonet/qsvm4sentanalysis/tree/main/subqubo}}.

\subsection{Main Method}

\texttt{QSplitSampler} distinguishes the execution of the main block into two categories: 
\begin{itemize} 
	\item Small-scale problems, or generally manageable-sized problems; 
	\item Problems that require partitioning. 
\end{itemize}

In the first case, the algorithm uses the QPU to solve QUBO instances where the problem size is less than \texttt{CUT\_DIM}, the cut-off size that distinguishes a manageable problem from one that cannot be addressed directly.

In the second case, the QUBO problem is partitioned using the \texttt{split\_problem} method, presented in Section \ref{sec:split}. 
Once the solutions are computed, they are aggregated using \texttt{aggregate\_solutions}, Section \ref{sec:aggr}.

The reference code can be found in Listing \ref{code:main}.

\begin{lstlisting}[language=Python, caption=QSplitSampler main function, label=code:main]
def QSplitSampler(qubo: QUBO, dim: int) -> QUBO:
    if dim <= CUT_DIM:
        res = SAMPLER.sample_qubo(qubo.qubo_dict, num_reads=10)
        qubo.solutions = res[min(res.energy)]
        return qubo

    sub_problems = split_problem(qubo, dim)
    for i, q in enumerate(sub_problems):
        sub_problems[i] = QSplitSampler(q, dim // 2)
    return aggregate_solutions(sub_problems, qubo)
\end{lstlisting}

For example, suppose setting the parameters of \texttt{QSplitSampler} as follows:
\begin{itemize}
    \item The problems solved directly via QPU are composed of two variables;
    \item The QPU returns four assignments;
    \item At the end of each recursive step, only the three best solutions are retained.
\end{itemize}

A starting QUBO problem could be:

\begin{equation}
    \begin{bmatrix}
        x_1 & x_2 & x_3 & x_4
    \end{bmatrix}
    \begin{bmatrix}
        1 & 2 & 3 & 4 \\
        0 & 5 & 6 & 7 \\
        0 & 0 & 8 & 9 \\
        0 & 0 & 0 & 10 \\
    \end{bmatrix}
    \begin{bmatrix}
        x_1 \\
        x_2 \\
        x_3 \\
        x_4
    \end{bmatrix}
    \label{eq:example_full}
\end{equation}

\subsection{Split Method}\label{sec:split}

The partitioning methodology follows the approach described in Section \ref{sec:prop}, where the matrix describing the QUBO problem is divided into four parts, with only three being returned, as one is zero by definition and does not contribute to the solution.

The order in which the new subproblems are returned is as follows: 
\begin{itemize} 
	\item The $\operatorname{UL}$ problem, corresponding to the top-left subproblem, it operates on the first half of the optimization variables; 
	\item The $\operatorname{UR}$ problem, the top-right sub-matrix portion, operating on both partitions of the variables; 
	\item The $\operatorname{BR}$ problem, the bottom-right sub-matrix portion, operates on the second half of the optimization variables. 
\end{itemize}

The following subproblems are obtained from \eqref{eq:example_full}:
\begin{itemize}
    \item $\operatorname{UL}$
    \begin{equation}
        \begin{bmatrix}
            x_1 & x_2
        \end{bmatrix}
        \begin{bmatrix}
            1 & 2 \\
            0 & 5
        \end{bmatrix}
        \begin{bmatrix}
            x_1 \\
            x_2
        \end{bmatrix}
        \label{eq:exampleUL}
    \end{equation}
    \item $\operatorname{UR}$
    \begin{equation}
        \begin{bmatrix}
            x_1 & x_2
        \end{bmatrix}
        \begin{bmatrix}
            3 & 4 \\
            6 & 7
        \end{bmatrix}
        \begin{bmatrix}
            x_3 \\
            x_4
        \end{bmatrix}
        \label{eq:exampleUR}
    \end{equation}
    \item $\operatorname{BR}$
    \begin{equation}
        \begin{bmatrix}
            x_3 & x_4
        \end{bmatrix}
        \begin{bmatrix}
            8 & 9 \\
            0 & 10
        \end{bmatrix}
        \begin{bmatrix}
            x_3 \\
            x_4
        \end{bmatrix}
        \label{eq:exampleBR}
    \end{equation}
\end{itemize}

Although all matrix portions are converted to be upper triangular, problem obtained by $\operatorname{UR}$ like \eqref{eq:exampleUR} is, by definition, not in QUBO form. 
This is because it operates on both partitions of the optimization variables, meaning $X^T \neq X$.

The compact representation required by D-Wave allows the solver to determine the problem dimensions automatically. 
Therefore, the problem solved in this case does not take the form shown in Equation \eqref{eq:ur}.
The subproblem \eqref{eq:exampleUR}, for example, takes the form in \eqref{eq:realur}.

\begin{equation}
    \begin{bmatrix}
        x_1 & x_2 & x_3 & x_4
    \end{bmatrix}
    \begin{bmatrix}
        0 & 0 & 8 & 9 \\
        0 & 0 & 0 & 10 \\
        0 & 0 & 0 & 0 \\
        0 & 0 & 0 & 0 
    \end{bmatrix}
    \begin{bmatrix}
        x_1 \\
        x_2 \\
        x_3 \\
        x_4 
    \end{bmatrix}
    \label{eq:realur}
\end{equation}

That is, the subproblem is ``embedded'' in a zero matrix with dimensions equal to those of the original problem. 
This operation seems to negate the dimensionality reduction of the QUBO problem; however, the underlying representation of the problem remains sufficiently small to allow its resolution like the other two sub-matrices.

\subsection{Aggregation Method}\label{sec:aggr}

Once the subproblems are solved, the results must be aggregated to provide a complete solution to the problem.

\paragraph{UL-BR aggregation} Since these subproblems operate on different partitions of the variables, a Cartesian product of their partial assignments is performed to consider all possible combinations. 

Referring $S_{\operatorname{UL}}$, Equation \eqref{eq:s_ul}, as the solutions of \eqref{eq:exampleUL}, and $S_{\operatorname{BR}}$, Equation \eqref{eq:s_br}, as the solutions of \eqref{eq:exampleBR}, from their Cartesian product the solutions given in \eqref{eq:ulrb} are obtained.

\begin{equation}
    S_{\operatorname{UL}} =\{\{x_1=0,x_2=0\}, \{x_1=1,x_2=0\}, \{x_1=0,x_2=1\}\}
    \label{eq:s_ul}
\end{equation}

\begin{equation}
    S_{\operatorname{BR}} =\{\{x_3=0,x_4=0\}, \{x_3=1,x_4=0\}\}
    \label{eq:s_br}
\end{equation}

\begin{equation}
    S_{\operatorname{UL}-\operatorname{BR}}=\left[
    \begin{array}{c}
        \{x_1=0,x_2=0,x_3=1,x_4=0\} \\
        \{x_1=1,x_2=0,x_3=0,x_4=0\} \\
        \{x_1=1,x_2=0,x_3=1,x_4=0\} \\
        \{x_1=0,x_2=1,x_3=0,x_4=0\} \\
        \{x_1=0,x_2=1,x_3=1,x_4=0\} \\
        \{x_1=0,x_2=0,x_3=0,x_4=0\}
    \end{array}
    \right]
    \label{eq:ulrb}
\end{equation}

\paragraph{UR contribution} For each assignment of $S_{\operatorname{UL}-\operatorname{BR}}$ \eqref{eq:ulrb} the closest assignment of $S_{\operatorname{UR}}$, Equation \eqref{eq:ursol}, is searched.

\begin{equation}
    S_{\operatorname{UR}}=\left[
        \begin{array}{c}
            \{x_1=0,x_2=0,x_3=0,x_4=0\},\\ 
            \{x_1=1,x_2=0,x_3=0,x_4=0\},\\
            \{x_1=0,x_2=1,x_3=0,x_4=0\},\\
            \{x_1=0,x_2=0,x_3=1,x_4=0\}
        \end{array}
        \right]
        \label{eq:ursol}
\end{equation}

The following pairs of assignments are obtained:
\begin{itemize}
    \item $S_1 = \langle S_{\operatorname{UL}-\operatorname{BR}}[1], S_{\operatorname{UR}}[4]\rangle$, with no conflicting variables;;
    \item $S_2 = \langle S_{\operatorname{UL}-\operatorname{BR}}[2], S_{\operatorname{UR}}[2]\rangle$, with no conflicting variables;;
    \item $S_3 = \langle S_{\operatorname{UL}-\operatorname{BR}}[3], S_{\operatorname{UR}}[2]\rangle$, conflicting on the variable $x_3$;
    \item $S_4 = \langle S_{\operatorname{UL}-\operatorname{BR}}[4], S_{\operatorname{UR}}[3]\rangle$, with no conflicting variables;;
    \item $S_5 = \langle S_{\operatorname{UL}-\operatorname{BR}}[5], S_{\operatorname{UR}}[3]\rangle$, conflicting on the variable $x_3$;
    \item $S_6 = \langle S_{\operatorname{UL}-\operatorname{BR}}[6], S_{\operatorname{UR}}[1]\rangle$, with no conflicting variables;.
\end{itemize}

\paragraph{Final answer set} The resolution of conflicting assignments is dealt with in detail in Section \ref{sec:qsearch}, once assignments without conflicting variables have been obtained, the value of the objective function $X^TQX$ is calculated and the $k$ assignments with the lowest value are kept.

Solving the conflicts in $S_3$ and $S_5$, the final assignments produced by \texttt{QSplitSampler} are:
\begin{itemize}
    \item $S_1$, with an objective function value of 8;
    \item $S_2 = S_3$, with objective function value of 1;
    \item $S_4 = S_5$, with objective function value of 5;
    \item $S_6$, with objective function value of 0.
\end{itemize}

The resulting assignments produced for the problem \eqref{eq:example_full} are then $S_6$, $S_2$ and $S_4$

\subsection{Quantum Local Search}\label{sec:qsearch}

Conflicting assignments can be resolved using various search methodologies, including: 
\begin{enumerate} 
	\item Brute-force; 
	\item Local search; 
	\item Extraction of a new QUBO problem. 
\end{enumerate}

Brute-force was initially implemented, but for non-trivial problems, the number of assignments to test led to intractable issues.

Local search techniques might be promising, but it would go against the principle underlying the development of \texttt{QSplitSampler}, which is to maximize the use of the QPU.

For these reasons, the most reasonable option was to assess the feasibility of using the QPU to resolve conflicting assignments.

\begin{lstlisting}[language=Python, caption=QUBO extractor function, label=code:qsearch]
def q_search(df: pd.DataFrame, qubo: QUBO) -> pd.DataFrame:
    for i, row in df.iterrows():
        no_energy = row.drop(energy)

        nans = no_energy[np.isnan(no_energy)]
        qubo_nans = defaultdict(int)
        for row_idx in nans.index:
            for col_idx in nans.index:
                val = qubo.qubo_dict.get((row_idx, col_idx), 0)
                qubo_nans[(row_idx, col_idx)] = val
        nans_sol = SAMPLER.sample_qubo(qubo_nans, num_reads=10)
        nans_sol = nans_sol.to_pandas_dataframe()
                           .sort_values(by=energy, 
                                        ascending=True).iloc[0]
        df.loc[i, nans.index] = nans_sol.drop(energy)
        df.loc[i, energy] += nans_sol.energy

    return df
\end{lstlisting}

The underlying idea of the proposed in Listing \ref{code:qsearch} search mechanism is to extract a subproblem focusing on the rows and columns associated with the conflicting variables. 
Once extracted and converted in upper triangular, the subproblem can be resolved again on the QPU.

In the problem \eqref{eq:example_full} the only variable that generates conflicting assignments is $x_3$, so the row and column relating to that variable must be extracted from the matrix $Q$.
Since there is only one value, the QUBO formulation reduces to $\min x_3*8*x_3$, which is minimised by assigning $x_3$ the value zero.

Unlike the method discussed in Section \ref{sec:split}, this search approach does not guarantee a specific size for the resulting problem.
Because in the case of complementary assignments, the number of conflicting variables equals the total number of variables considered. 

Experimentally, the number of variables with conflicting assignments remained small enough to allow resolution via the QPU.