\chapter{Results}

\section{Classification Methodologies Compared}\label{sec:qsvm-res}

The analysis focuses on comparing three fundamental aspects that characterize each machine learning model:
\begin{enumerate}
    \item The quality of the classification;
    \item The time required for the training phase;
    \item And the time needed to classify new examples.
\end{enumerate}

In the subsequent analyses, times will be reported in seconds. Since both training and inference are considered waiting times: 
\begin{itemize}
    \item The former for obtaining the actual model;
    \item The latter for obtaining the relevant data.
\end{itemize}

The best result will be closest to zero, i.e., the procedure that takes the least time to provide the desired result.

Evaluating performance is not as straightforward because a wrong choice of judgment method may highlight secondary flaws and conceal potential strengths. Conversely, it could lead to the opposite situation, concluding that the proposed approach is better than it should be.

As this is a binary classification task, classic machine-learning metrics can be utilized. The data obtained from the inference phase are divided based on the original labels of the dataset, assigning the \emph{Actual Positive} (AP) set to the data labelled $1$ and the \emph{Actual Negative} (AN) set to the data labelled $-1$. Test data are also divided based on the class produced by the model, generating the \emph{Predicted Positive} (PP) and \emph{Predicted Negative} (PN) sets. Based on these four sets, the \emph{contingency table}, Table \ref{tab:contingency}, is determined.

\begin{table}[H]
    \centering
    \begin{tabular}{l|ll}
       & PP & PN \\\hline
    AP & TP & FN \\
    AN & FP & TN
    \end{tabular}
    \caption{Contingency table}
    \label{tab:contingency}
\end{table}

The intersection between two of the above-described sets defines:
\begin{itemize}
    \item \emph{True Positive} (TP), the elements of the positive class categorised correctly;
    \item \emph{False Positive} (FP), elements of the negative class that were classified as positive;
    \item \emph{True Negative} (TN), elements correctly classified as negative;
    \item \emph{False Negative} (FN), elements erroneously labelled as negative.
\end{itemize}

From the values in the contingency table, various evaluation metrics can be derived, all of which are within the interval $[0, 1]$, where $0$ indicates the worst possible value and $1$ is the perfect score.

\paragraph{Accuracy} $\frac{TP + TN}{TP + TN + FP + FN}$, indicates the number of correct predictions out of the total available examples, thus not providing explanations for the errors obtained.

\paragraph{Precision} Derived using the formula $\frac{TP}{TP + FP}$, it indicates how many of the elements labelled as positive should have been positive.

\paragraph{Recall} From the formula $\frac{TP}{TP + FN}$, it provides a measure of how well a relevant datum (positive class) is correctly interpreted by the model.

\paragraph{F1-Score} The formula is obtained through the harmonic mean of precision and recall, $2*\frac{\text{Precision}*\text{Recall}}{\text{Precision}+\text{Recall}}$. This metric avoids degenerate situations where a desirable result is obtained on one of the two metrics while significantly decreasing the score of the other.

These four metrics will be considered for evaluating the obtained models, to capture as much information as possible for assessing performance during the classification of new examples.

\section{Performance Analysis}\label{sec:qsvm-performance}

Using the APIs provided by D-Wave, it is possible to calculate the minimum time required by a CSP problem, Section \ref{sec:qadvantages}, to be solved by the hybrid solver available.

To tackle the optimisation problem generated by TweetDF, Section \ref{sec:tweetdf}, the expected calculation time is approximately 480 seconds, due to the number of optimisation variables $\alpha$ in Equation \eqref{eq:min-svm}, each one associated with an example of the training set. 
This calculation time is excessively high compared to the free version of D-Wave's hybrid resolution services, which are limited to 20 minutes.

To conduct a larger number of iterations and extract the average behaviour, the size of the dataset can be reduced, allowing for faster resolution. 
An appropriate compromise between the number of examples and the time required by the hybrid solver was obtained using a dataset of 4096 examples for the model training phase. 
These examples were randomly selected from TweetDF, ensuring an equal number of elements of the positive and negative classes. 
In contrast, the number of examples for the inference phase was reduced to 2048 elements.

The subset extraction of the data was performed randomly, only imposing an equal number of positive and negative examples.

The results obtained from ten distinct executions are reported in Tables \ref{tab:QSVM1}, \ref{tab:QSVM2}, and \ref{tab:QSVM3}. These tables are presented to allow a comparison with classical methods for solving the same classification task, represented by CPLEX and RoBERTa.

\paragraph{Classical Optimizer CPLEX} The CPLEX program\cite{cplex} is an optimization software for solving linear and non-linear programming problems. The mathematical formulation of the CSP describing an SVM is represented using the PyOmo library, which allows abstraction concerning the chosen solver and the generic writing of problems. Once the problem is formulated as an objective function and constraints, calls to the optimization library are made, returning the assignment of the variables $\alpha$ that minimizes the objective function \eqref{eq:min-svm}. The results regarding the resolution of the problem through CPLEX are detailed in Table \ref{tab:QSVM1}.

\begin{table}[H]
    \centering
    \begin{tabular}{ccccccc}
    \toprule                                                             \\                    
    Train time & Accuracy & Precision & Recall & F1    & Time to predict \\
    \midrule
    285.207    & 0.8      & 0.95      & 0.64   & 0.77  & 2.324           \\
    278.999    & 0.83     & 0.91      & 0.74   & 0.81  & 1.997           \\
    267.266    & 0.81     & 0.97      & 0.65   & 0.77  & 2.059           \\
    256.59     & 0.8      & 0.96      & 0.64   & 0.76  & 2.353           \\
    254.312    & 0.78     & 0.95      & 0.59   & 0.73  & 2.105           \\
    271.298    & 0.81     & 0.93      & 0.68   & 0.79  & 2.383           \\
    271.301    & 0.79     & 0.91      & 0.64   & 0.75  & 2.392           \\
    266.922    & 0.82     & 0.93      & 0.69   & 0.79  & 2.148           \\
    257.671    & 0.79     & 0.96      & 0.61   & 0.75  & 2.398           \\
    264.427    & 0.8      & 0.88      & 0.68   & 0.77  & 1.977           \\
    \bottomrule
    \end{tabular}
    \caption{Results obtained through CPLEX}
    \label{tab:QSVM1}
\end{table}

\paragraph{Hybrid Optimizer from D-Wave} D-Wave offers a suite of hybrid solvers, for optimization problems, available through Python APIs\cite{hss}. The program is always initialized using the PyOmo library and then transferred to D-Wave's cloud services for resolution. Unlike CPLEX, which returns a single optimal solution, the solver provided by D-Wave produces a list of possible assignments, each with its associated objective function value. By extracting all and only the assignments with the minimum objective function value, it is possible to produce a set of models and perform the inference phase by aggregating the individual results through a majority vote. Although this procedure slows down the prediction, it could improve the inference phase. The results obtained using the solvers proposed by D-Wave are detailed in Table \ref{tab:QSVM2}.

\begin{table}[H]
    \centering
    \begin{tabular}{ccccccccc}
    \toprule
    Train time & Accuracy & Precision & Recall & F1    & Time to predict \\
    \midrule
    390.745    & 0.81     & 0.97      & 0.64   & 0.77  & 40.219          \\
    423.917    & 0.79     & 0.96      & 0.6    & 0.74  & 30.186          \\
    326.422    & 0.79     & 0.97      & 0.59   & 0.73  & 23.686          \\
    239.25     & 0.76     & 0.95      & 0.55   & 0.7   & 19.74           \\
    221.635    & 0.81     & 0.96      & 0.65   & 0.77  & 39.708          \\
    320.892    & 0.8      & 0.96      & 0.63   & 0.76  & 34.159          \\
    319.161    & 0.81     & 0.96      & 0.65   & 0.78  & 40.224          \\
    437.813    & 0.82     & 0.97      & 0.67   & 0.79  & 40.424          \\
    377.309    & 0.81     & 0.97      & 0.65   & 0.78  & 30.541          \\
    220.48     & 0.82     & 0.97      & 0.66   & 0.79  & 39.654          \\
    \bottomrule
    \end{tabular}
    \caption{Results obtained through D-Wave}
    \label{tab:QSVM2}
\end{table}

\paragraph{The Transformer Model RoBERTa} In addition to the optimization methods reported, it is relevant to evaluate the performance of the D-Wave hybrid solvers compared to the state-of-the-art. The model considered belongs to the BERT family. The foundational version of RoBERTa\cite{roberta-base} was trained on the Sentiment Analysis task using datasets of various natures, taking approximately 15 days to converge. Subsequently, the model was fine-tuned on TweetEval\cite{tweetRoberta}\cite{robertamodel} to specialize it in classifying ``basic'' sentiments.

\begin{table}[H]
    \centering
    \begin{tabular}{ccccc}
    \toprule
    Accuracy & Precision & Recall & F1    & Time to predict \\
    \midrule
    0.95     & 0.95      & 0.94   & 0.95  & 142.569         \\
    0.94     & 0.94      & 0.94   & 0.94  & 124.094         \\
    0.94     & 0.94      & 0.93   & 0.94  & 141.828         \\
    0.95     & 0.95      & 0.94   & 0.95  & 133.949         \\
    0.94     & 0.95      & 0.93   & 0.94  & 147.302         \\
    0.94     & 0.95      & 0.93   & 0.94  & 135.804         \\
    0.94     & 0.95      & 0.93   & 0.94  & 136.883         \\
    0.94     & 0.95      & 0.93   & 0.94  & 132.836         \\
    0.95     & 0.96      & 0.94   & 0.95  & 137.382         \\
    0.94     & 0.95      & 0.94   & 0.94  & 135.044         \\
    \bottomrule
    \end{tabular}
    \caption{Results obtained through RoBERTa}
    \label{tab:QSVM3}
\end{table}

Considering the results reported in different contexts, the following analysis is done of the three characteristics of interest identified in the Section \ref{sec:qsvm-res}.

\paragraph{Performance during Inference} The results recorded by RoBERTa are significantly better compared to those obtained from SVMs, both in the quantum version and the CPLEX classical one.
The first score recorded by CPLEX, Table \ref{tab:QSVM1}, shows 0.77 as F1, the same score is obtained by the hybrid D-Wave solver, Table \ref{tab:QSVM2}, while on the same dataset RoBERTa obtains an F1 score of 0.95, thus increasing by more than 20\%.

Despite this, the machine learning models also report scores well above a random guess, leading to the conclusion that the proposed methodology can capture the semantic information conveyed by the text, although not as well as the Transformer architecture, which is renowned in the literature for its superior ability to convey textual information. No significant differences are noted between the model produced by D-Wave and the one generated with CPLEX, indicating that the reduction in the optimization variable domain, discussed in \ref{sec:domain}, does not impact model performance.

\paragraph{Training Time} Although unavailable, the training times for an expressive and fine-tuned model like RoBERTa are certainly higher than those recorded by the optimization approaches. 
At the very least, it is reasonable to expect times in the order of hours, if not days, on high-performance GPUs. 
Conversely, the D-Wave approach and the one based on CPLEX allowed training the model on a personal computer. 
The recorded times, between 250 and 450 seconds, consider several external factors such as:
\begin{itemize}
    \item The need to convert the data into a suitable format to be processed by the optimisation procedure, which takes about 80 seconds;
    \item For D-Wave optimizer, sending the model to the cloud infrastructure, while it is not possible to record the time taken to transmit the data, one can consider the size of the CSP on disk, which is around 4GB.  
\end{itemize} 
Analyzing only the time required to produce the response to the problem, the CPLEX optimizer takes an average of 100 seconds, while D-Wave's hybrid computation produces comparable quality responses in only 40 seconds, significantly reducing the required time.

\paragraph{Classification Time} When inferring new examples, the times recor\-ded by RoBERTa were the worst of the three methodologies compared. 
Although in this case, the unit of measurement is still seconds, the time required to classify the same number of new examples is significantly greater. 
Among the SVM variants, classical and quantum, the version produced by D-Wave is less performant than that obtained with CPLEX. 
This behaviour is due to the use of multiple models to calculate the result, a procedure not performed with the result obtained from CPLEX. 

\subparagraph{Improving D-Wave Classification Time} From a qualitative analysis of the ensemble procedure, using multiple models does not seem to bring particular benefits; In almost all cases, the majority class equals the class predicted by each model in the set. 
For this reason, it is possible to avoid the computational overhead of the ensemble procedure and use only one of the optimal models obtained through D-Wave, which reduces inference time, making the results produced by CPLEX and D-Wave comparable.

\section{Testing with the whole dataset}

Since the results obtained from different executions are almost identical, it is possible to perform a single test using the entire TweetDF. This additional test allows verification of the consistency of performance across the entire data set and the scalability of the hybrid optimisation.

\begin{table}[H]
    \centering
    \begin{tabular}{cccccc}
    \toprule
    \multicolumn{6}{c}{D-Wave}                                       \\ \midrule
    Train (min) & Accuracy & Precision & Recall & F1   & Predict (s) \\
    35          & 0.82     & 0.96      & 0.66   & 0.78 & 212.3       \\ \bottomrule
    \toprule
    \multicolumn{6}{c}{RoBERTa}                                      \\ \midrule
                & Accuracy & Precision & Recall & F1   & Predict (s) \\
                & 0.94     & 0.95      & 0.94   & 0.94 & 291         \\ \bottomrule
    \end{tabular}
    \caption{Results with whole TweetDF}
    \label{tab:qsvm-fulldf}
\end{table}

The results shown in Table \ref{tab:qsvm-fulldf} pertain to RoBERTa and D-Wave. The data regarding CPLEX are not reported since the SVM problem with such a large dataset caused the optimizer to crash before producing a result. This incident highlights a significant advantage of quantum computing: the ability to handle not only more complex problems but also larger ones. This is relevant because, in general, the more information encoded in the problem, the better the optimal solution produced.

\paragraph{Performance during Inference} The performance results do not vary significantly, indicating that a random subset of TweetDF is sufficient to capture the semantic information that can be modelled with SVMs.

\paragraph{Training Time} The training time using D-Wave rises to 35 minutes, of which 8 minutes are used on the D-Wave hybrid solver. 
Despite the considerable increase in training times, the optimization conducted by D-Wave is still much shorter than what is reasonable to expect for training RoBERTa.

\paragraph{Classification Time} The simultaneous use of several models by D-Wave has a significant impact on the time required to infer the class of new examples. While RoBERTa's inference times are even longer, the difference is insufficient to justify an architectural change. By using only one of the equivalent models presented by D-Wave's hybrid solver, it would be possible to reduce the calculation time and generate answers more efficiently.

\section{QSVM Overview}

In summary, machine-learning approaches conducted with CPLEX and D-Wave showed:
\begin{itemize}
    \item To converge quickly to the optimal solution, requiring a small dataset size and not taking advantage of the increase in input data;
    \item To approximate Transformer performance in contexts where time and computational power constraints are involved, such as embedded systems and personal computers.
\end{itemize}

Specifically, the approach leveraging quantum computing (D-Wave) has proven to be superior to classical optimization (CPLEX), allowing for solving the presented problems in less time and with comparable performance.
