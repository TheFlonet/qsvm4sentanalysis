\documentclass[aspectratio=169]{beamer}
\usepackage{graphicx}
\usepackage[italian]{babel}
\usepackage{hyperref}

\usetheme[progressbar=frametitle]{metropolis}
\title{Quantum Approaches to Sentiment Analysis}
\author{\emph{Candidate}: Mario Bifulco}
\date{\emph{Supervisor}: Luca Roversi}
\institute{University of Turin}

\begin{document}
\setbeamertemplate{section in toc}[sections numbered]

\begin{frame}
    \titlepage
\end{frame}

%\begin{frame}
%    \tableofcontents
%\end{frame}

\section{Quantum Support Vector Machine for Sentiment Analysis}

\begin{frame}
    \frametitle{Why use non-classical architectures?}

    

\end{frame}

\begin{frame}
    \frametitle{Why Support Vector Machine?}

    

\end{frame}

\begin{frame}
    \frametitle{Empirical comparison}

    

\end{frame}

\begin{frame}
    \frametitle{Qualitative analysis}

    

\end{frame}

\section{Balancing QPU and CPU execution time}

\begin{frame}
    \frametitle{Usage of QPU}

    

\end{frame}

\begin{frame}
    \frametitle{Why moving the computation to QPU?}

    

\end{frame}

\begin{frame}
    \frametitle{Turning a problem from CSP to QUBO}

    

\end{frame}

\section{Developing an hybrid solver}

\begin{frame}
    \frametitle{The problem of the minor embedding algorithm}

    

\end{frame}

\begin{frame}
    \frametitle{Embedding Search Times}

    

\end{frame}

\begin{frame}
    \frametitle{Algebric decomposition of QUBO problems}

    

\end{frame}

\begin{frame}
    \frametitle{Results}

    

\end{frame}

\begin{frame}
    \frametitle{Future works}

    

\end{frame}

\end{document}